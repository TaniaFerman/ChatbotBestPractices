\documentclass[a4paper,10pt]{article}
\usepackage[utf8]{inputenc}
\usepackage{graphicx}
\usepackage{booktabs}
\usepackage{xcolor}
\title{Chatbot's Best Practices List}
\author{Maria Ferman}

\begin{document}

\maketitle
%GRAMMAR!
\section{Chatbot Purpose}

Designers like Christ should give an specific purpose to the chatbot. So when Laura uses the chatbot for the first time, she will understand what the chatbot was created.

\section{Conversational and Situational Knowledge}

In order to have a good conversation the chatbot and the user should share a conversational and situational knowledge.  Conversational knowledge is when Laura should be able to follow the meaning of the chatbot remarks, while the chatbot should be able to select a response that fits coherently into what Laura is saying. Situaltional knowledge is when the chatbot should be situation-aware regarding Laura’s context.

\section{Chatbot Interactions}

Conversational scripts allow Chris to have a draft outlining the situations and actions the chatbot need to support. Those scripts determine the limitation of the chatbot by having specific scenarios about the conversation. 

\section{Interaction Elements}

As in a regular conversation, a conversation with a chatbot may contain:

Assertions: Informational messages, Error Messages

Questions: Verification Questions, Open Questions(uncover context), Close Questions(confirm understanding)

Support: Help, Data Insights.

\begin{table}[]
\centering

\label{InteractionElementsTable}
\begin{tabular}{lllll}
\hline
\textbf{Assertions}    & \textbf{Questions}     & \textbf{Support}   \\
\hline
Informational messages & Verification Questions & Help      \\
Error Messages         & Open Questions         & Data Insights.  \\
                       & Close Questions        &       \\
     \hline                   
\end{tabular}
\caption{Interaction elements}
\end{table}

 Besides regular human conversational elements, chatbots may have command-line statements (expressions)*. These statements are commands that users with a domain specific knowledge use to interact with the chatbot.
 
 \section{Chatbot Language}
 
 Chatbots should use the same language and words as the user. The use of technical and system-oriented terms should be according to the audience. 


\medskip

\bibliographystyle{unsrt}%Used BibTeX style is unsrt
\bibliography{References.bib}

\end{document}