\documentclass[a4paper,10pt]{article}
\usepackage[utf8]{inputenc}
\usepackage{graphicx}
\usepackage{booktabs}
\usepackage{xcolor}
\title{Chatbot's Best Practices List}
\author{Maria Ferman}

\begin{document}

\maketitle
%GRAMMAR!
\section{Chatbot Purpose}

Designers like Christ should give an specific purpose to the chatbot. So when Laura uses the chatbot for the first time, she will understand why the chatbot was created.

\section{Conversational and Situational Knowledge}

In order to have a good conversation the chatbot and the user should share a conversational and situational knowledge.  Conversational knowledge is when Laura should be able to follow the meaning of the chatbot remarks, while the chatbot should be able to select a response that fits coherently into what Laura is saying. Situaltional knowledge is when the chatbot should be situation-aware regarding Laura’s context.

\section{Chatbot Interactions}

Conversational scripts allow Chris to have a draft outlining the situations and actions the chatbot need to support. Those scripts determine the limitation of the chatbot by having specific scenarios about the conversation. 

\section{Interaction Elements}

As in a regular conversation, a conversation with a chatbot may contain:

Assertions: Informational messages, Error Messages

Questions: Verification Questions, Open Questions(uncover context), Close Questions(confirm understanding)

Support: Help, Data Insights.

\begin{table}[]
\centering

\label{InteractionElementsTable}
\begin{tabular}{lllll}
\hline
\textbf{Assertions}    & \textbf{Questions}     & \textbf{Support}   \\
\hline
Informational messages & Verification Questions & Help      \\
Error Messages         & Open Questions         & Data Insights.  \\
                       & Close Questions        &       \\
     \hline                   
\end{tabular}
\caption{Interaction elements}
\end{table}

 Besides regular human conversational elements, chatbots may have command-line statements (expressions)*. These statements are commands that users with a domain specific knowledge use to interact with the chatbot.
 
 \section{Chatbot Language}
 
 Chatbots should use the same language and words as the user. The use of technical and system-oriented terms should be according to the audience. 

\section{Chatbot Conversation Flow}

The conversational chatbot flow should follow a certain order. This means that after a particular information was given, the chatbot should perform a certain task. Designers should think about the different possible flows that the user may go in and the different possible flows that the chatbot may use.  

\section{User's Control}

Chatbots should not make users believe that they are powerless; they are there to help users.

Chatbots should be guides or an additional aid that allows users to understand a process and have better results.

\section{Making Changes on the Fly}

Chatbots should have the ability to undo or redo the responses that the user gave.

\section{User's Recognition and Recall}

Users do not like to read a large amount of text. Therefore, chatbots should keep their messages short and give the users a small number of options, in order to avoid getting them lost or feeling overwhelmed. This will allow to have a minimal design and a better user comprehension.

\section{User's input}

When designers need to provide different options the use of a carousel is the most efficient way to do it. When designers need to ask a closed-ended question, the chatbot may give a yes or no button, in order to allow users to answer the question. When users need to type an answer, designers need to keep in mind that users will not type a large amount of text, especially those that are using the chatbot through a mobile phone. Therefore, all the questions should be intended on having a not long answer from the user. 

\section{Chatbot's Personality}

A chatbot’s personality is what makes it unique. By adding extra information to the dialogue and having a consistent language, the chatbot can have a distinctive personality. When designers are selecting the chatbot personality, they must consider the chatbot audience and the task that the chatbot is trying to support.

\section{Chatbot's Memory}

Chatbots may be built with the ability to keep track of the user’s interest and preferences, even if they change over time (situational knowledge).  The user’s habits and preferences can be saved in the different sessions that he or she had with the chatbot.

\section{Chatbot's Communication Capabilities}

Chatbots should have a delimited script of the possible scenarios. If a user wants to go through an unknown topic for the chatbot, the chatbot has to specify to the user that this question goes beyond its knowledge.

\section{Chatbot's /help}

A chatbot appropriate documentation should include a concrete list of what steps can be carried out and a description of the chatbot’s capabilities.

\medskip

\bibliographystyle{unsrt}%Used BibTeX style is unsrt
\bibliography{References.bib}

\end{document}