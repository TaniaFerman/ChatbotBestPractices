\documentclass[a4paper,10pt]{article}
\usepackage[utf8]{inputenc}

\title{Chatbot Design Principles}
\author{Maria Ferman}

\begin{document}

\maketitle

Nowadays, Software is getting more and more customized and at the same time, it is necessary to make software accessible for (to) users that do not have any technological background. Fortunately, chatbots can help with both issues. Chatbots allow users to interact and communicate with software by using their own words. Therefore, software can become easier to understand and use. Chatbots also allows to save the user preferences. By doing this, software can adapt to the interest of the users.   

\subsection*{Chatbot Language}

Chatbots should use the same language and words as the user, and avoid to use technical and system-oriented terms. Chatbots can mimic how users normally speak, by designing a chatbot oriented to certain populations. Therefore, before the creation of the chatbot, it is necessary to have a "solid understanding of the audience we seek to appeal to" \cite{HeuristicsWebPage}.  

\subsection*{User's Control}

Users are used to and likes to have control over a software. Therefore, it is extremely important to make the user feel that he or she has the control over the interface. Chatbots should not make users believe that they are powerless over the interface, chatbots are there to help users to have a better experience using the interface. Chatbots should be guides or an additional aid that allow users to understand the interface and have better results. According to Sheneiderman \cite{shneiderman1997direct} it is "necessary to give the users the feeling of being in control and therefore they can be responsible for the decisions they make.  

\subsection*{User's Freedom}

It is normal that users make mistakes when they are choosing different options in an interface. Chatbots should have the ability to undo or redo the changes that the user just selected. In that case, users know they are able to undo an unwanted change triggered by an misinterpreted message or a mistaken click. 

\medskip

\bibliographystyle{unsrt}%Used BibTeX style is unsrt
\bibliography{References.bib}

\end{document}