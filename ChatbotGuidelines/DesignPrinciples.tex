\documentclass[a4paper,10pt]{article}
\usepackage[utf8]{inputenc}

\title{Chatbot Design Principles}
\author{Maria Ferman}

\begin{document}

\maketitle

Nowadays, software is getting more and more customized and at the same time, it is necessary to make software accessible for users that do not have any technological background. Fortunately, chatbots can help with both issues. Chatbots allow users to interact and communicate with software by using their own words. Therefore, software can become easier to understand and use. Chatbots also allow the interface to save the user preferences. By doing this, software can adapt to the interest of the users.  

\subsection*{Chatbot's purpose}

The purpose of the chatbot should be clear to the users. A chatbot that makes a user feel confused * about its purpose is futile *. Users need to have a clear understanding of why the bot was designed and what are its capabilities and limitations. 

\subsection*{Conversational Shared Knowledge}

Conversations need to follow certain rules and to have a certain semantic order for the sake of having a clear understanding between participants. Therefore, for having a meaningful conversation between chatbots and users, the chatbot script should follow the same rules and semantic order. The user should be able to follow the meaning of the chatbot remarks, meanwhile the chatbot should be able to select a response that fit coherently into what the user is saying. Therefore, is vital to understand how chatbots and users follow each other thoughts.

According to Reichman \cite{reichman1985getting} people need to share situational knowledge and semantic reference in order to understand and follow a conversation.\\[0\baselineskip]

\textbf{1.-  Common situational knowledge}\\[0\baselineskip]
In order to have a mutual understanding in a conversation, it is necessary that participants share a situational context. Situational Context involves previous knowledge about the topic that is being discussed.  Therefore, the user and the chatbot should \textit{share the same information ``for being on the same page."} 

Alexa, Amazon's voice control system, is a clear example of the importance of context for conversational agents. When  a San Diego TV station was covering the report of a girl ordering a dollhouse by mistake, the viewers' Alexa devices also recognized the command for purchase a dollhouse. ``If Alexa has recognized the context of the news programme (the TV show presenter speaking in a past tense), then Alexa would not have taken action" \cite{WhatwecanlearnfromAlexasmistakes}.\\[0\baselineskip]
% how much and what kind of detail is (and is not) stated.  

\textbf{2.-  Common semantic reference} \\[0\baselineskip]
Another main point is to share the same semantic knowledge. Participants in the conversation, not only need to understand the meaning of the speaker's words, but also \textit{the particular meanings those words take on in the immediate semantic context} \cite{reichman1985getting}. 
Listeners must be able to follow the context of the conversation; they need to grasp what is the point of what the speaker is saying right now and what he or she said before. Listeners need to understand the prior and the current context of the conversation to avoid being confused and to grasp the conversation's assertion. At the same time, speakers need to follow an order, in the conversation, for making their remarks easily understandable to the listeners. (a conversational order)*
% if the listener is not capable to grasp the utterance in the conversation, the listener will be lost and confuse, and he or she will need to ask clarification to the speaker. "What does this have to do with what we were talking about? 

\subsection*{Chatbot Language}
According to Allen et. al. \cite{allen1978conversation} in order to have a meaningful conversation is necessary to share a language and vocabulary in common. Chatbots should use the same language and words as the user, and avoid using technical and system-oriented terms. If the design of the conversational agent is oriented to certain populations, chatbots can mimic how users normally speak.  Therefore, before the creation of the chatbot, it is necessary to have a ``solid understanding of the audience we seek to appeal to, and to have a vocabulary familiar to them.”  \cite{HeuristicsWebPage}. 

\subsection*{Chatbot conversation flow}
According to Reichman \cite{reichman1985getting} in a usual conversation many details are being shared. To avoid an incoherent conversational flow, it is necessary that listeners understand when and why the conversational topic have been changed. For instance chatbots can twist the conversation from talking about a certain graph and when the user has selected the graph, turn the conversation into what color is better to use. Shifting the conversation topic should be made carefully to avoid misunderstandings. 

Chatbots need to follow a structured development of the conversation (by avoiding references that are not clear to the participants) and engage users into the conversation. Besides, the conversational flow should take into account the following elements*: timing, being an active participant and an active listener, and avoid over sharing information. 
The conversation between the user and the chatbot should not lead to any ambiguity about what to say in the conversation.Therefore, the chatbot's conversation should use structured messages that are clear and concise. The flow of the conversation should be natural and evident*

However, chatbots should be careful of not losing their conversational feeling*. Some chatbots only use structured messages, for this reason they seem to be a command line, instead of giving the user the conversational feeling that chatbots should have.  

%http://www.wikihow.com/Have-a-Great-Conversation

\subsubsection*{Chatbot conversational content}

According to Allen et. al. \cite{allen1978conversation} the speaker can emit in a conversation several verbal acts (elements?), such as assertions, questions, support, among others. Chatbots can emit the same acts. Assertions are affirmations that the chatbot give to the user; assertions determine the content of the conversation and allows to the listener to increase their knowledge about a certain topic. Questions, can be used to make sure that the provided information is the required by the user. Finally, support is any extra help that the chatbot is given to the user. Besides, chatbots can also give feedback as a part of the conversational support. 

\subsection*{User's Control}

Users are accustomed to and like having control over a software. Therefore, it is extremely important to make the user feel that he or she has the control over the interface. Chatbots should not make users believe that they are powerless over the interface; they are there to help users to have a better experience using the interface. Chatbots should be guides or an additional aid that allows users to understand the interface and have better results. According to Sheneiderman \cite{shneiderman1997direct} it is ``necessary to give the users the feeling of being in control and therefore they can be responsible for the decisions they make.” 

\subsection*{User's Freedom}

It is normal that users make mistakes when they are choosing different options in an interface. Chatbots should have the ability to undo or redo the changes that the user just selected. In that case, users know they are able to undo an unwanted change triggered by an misinterpreted message or a mistaken click \cite{HeuristicsWebPage}.The weather forecast conversational agent Poncho has the ability of asking users if the information provided was appropriate; if not, users can ask again for the right location \cite{poncho2017}. In addition, the users should be able to ask for extra information at any point of the conversational flow.


\subsection*{Error Messages}

Chatbots should express error messages in a straightforward way. They should not use codes or technical words in order to avoid confusing and stressing the user. Chatbots should be able to express that an error has occurred and add suggestions about how to solve or undo that error.  

\subsection*{User's Recognition and Recall}

According to Scott, users do not like to read a large amount of text : ``they will read the first message and then their eyes glaze over. They skim the rest of the text” \cite{HeuristicsWebPage}. Besides, the principle of least effort, set forth Zipf, specifies that people use shortened words and expression in a speech, in order to obtain the maximum communication by using the least cost \cite{allen1978conversation}. Therefore, chatbots should keep their dialogues short and give the users a small amount of options, in order to avoid users getting lost or feel overwhelmed.

In addition, users do not remember details of the options given by an interface. When users avoid reading a large amount of text, they may misunderstand the chatbot’s message and finish with an unsuccessful result. The design of the chatbot dialogue should only have relevant information to help users with their tasks and prevent a dull interactions with the chatbot. 
%... p.8

\subsection*{Chatbot's Personality}

Chatbots’ personality is what it makes a chatbot different. By adding extra information into the dialogue and having a matching language, the chatbot can have a distinctive personality. However, designers should be careful to not include too much additional information that makes the user feel bored or annoyed. 

The chatbot’s personality should take into account the target audience for which it was designed. The personality should match with the users: therefore, how friendly, sarcastic or humorous should only depend on the people who will use the chatbot. Users like to interact with chatbots in a human way \cite{HeuristicsWebPage}. For this reason designers should consider how to engage users into the conversation. 

According to Scott \cite{HeuristicsWebPage}, there is a difference between the content (relevant information to help the user) and the medium (the chatbot’s personality). Users need to be entertained and at the same time be helped. The chatbot’s success may be defined by how designers balance between entertainment and guidance.

\subsection*{Chatbot's versatility}

Chatbots should be versatile enough to be able to understand open questions and commands written by the user. If a user wants to use a command in the interaction with the chatbot, the dialogue should continue smoothly without breaking the flow.

\subsection*{Chatbot's memory?}

Chatbots have the capacity of keeping track of the users interest and preferences, even if they change over time. Conversational agents can save the user's habits and preferences in the different sessions that he or she had with the chatbot \cite{shneiderman1997direct}. As an example of this, Poncho can save the location of a certain user. Therefore, the next time that the user asks for a weather broadcast, Poncho already know which location to look at, \cite{poncho2017} saving the user time and keystrokes. Besides, this allow the user to have a customised experience. 

%If I come back, Poncho will remember my location, saving me a few keystrokes.*

\subsection*{Chatbot's documentation}

Chatbots should be used without the need of extra information. How to use a chatbot should be a simple conversational flow. However, there are some users that like to deeply understand how the interface works and detect its limitations. A chatbot appropriate documentation should be precise and short. It should provide extra information about how the chatbot can help users to focus their tasks. Besides, it also ``should include a concrete list of what steps can be carried out" and the critical points of the chatbot \cite{HeuristicsWebPage}. Finally, the chatbot should provide an easy way to access the documentation.   

\subsection*{Chatbot's limitations}

Chatbots should have a delimited script of the possible scenarios. If a user wants to go through an unknown topic for the bot, the bot has to specify to the user that that* question goes beyond its knowledge. For example, when a user asks to Poncho, the weather chatbot, something that it wasn't created to *, Poncho replies with: ``Oops, I didn't catch that. For things I can help you with, type help" \cite{HeuristicsWebPage}. Chatbots needs to gracefully establish limits to users that want to go beyond the bot's knowledge.  

\medskip

\bibliographystyle{unsrt}%Used BibTeX style is unsrt
\bibliography{References.bib}

\end{document}