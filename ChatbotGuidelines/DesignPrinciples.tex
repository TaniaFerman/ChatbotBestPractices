\documentclass[a4paper,10pt]{article}
\usepackage[utf8]{inputenc}

\title{Chatbot Design Principles}
\author{Maria Ferman}

\begin{document}

\maketitle

Nowadays, Software is getting more and more customized and at the same time, it is necessary to make software accessible for (to) users that do not have any technological background. Fortunately, chatbots can help with both issues. Chatbots allow users to interact and communicate with software by using their own words. Therefore, software can become easier to understand and use. Chatbots also allows to save the user preferences. By doing this, software can adapt to the interest of the users.   

\subsection*{Chatbot Language}

Chatbots should use the same language and words as the user, and avoid to use technical and system-oriented terms. Chatbots can mimic how users normally speak, by designing a chatbot oriented to certain populations. Therefore, before the creation of the chatbot, it is necessary to have a "solid understanding of the audience we seek to appeal to" \cite{HeuristicsWebPage}.  

\subsection*{User's Control}

Users are used to and likes to have control over a software. Therefore, it is extremely important to make the user feel that he or she has the control over the interface. Chatbots should not make users believe that they are powerless over the interface, chatbots are there to help users to have a better experience using the interface. Chatbots should be guides or an additional aid that allow users to understand the interface and have better results. According to Sheneiderman \cite{shneiderman1997direct} it is "necessary to give the users the feeling of being in control and therefore they can be responsible for the decisions they make."  

\subsection*{User's Freedom}

It is normal that users make mistakes when they are choosing different options in an interface. Chatbots should have the ability to undo or redo the changes that the user just selected. In that case, users know they are able to undo an unwanted change triggered by an misinterpreted message or a mistaken click \cite{HeuristicsWebPage}. Besides, the users should be able to ask for extra information at any point of the conversational flow. 

\subsection*{Error Messages}

Chatbots should express error messages in a straightforward way. They should avoid codes or technical words in order to avoid* confusing and stressing the user. Chatbots should be able to express that an error has occurred and add suggestions about how to solve or undo that error.  

\subsection*{User's Recognition and Recall}

According to Scott, users do not like to read a large amount of text, "they will read the first message and then their eyes glaze over. They skim the rest of the text" \cite{HeuristicsWebPage}. Therefore, chatbot should keep their dialogues short and give the users to select between a small amount of options, in order to avoid users to get lost or feel overwhelmed. Besides, users do not remember details of the options given by an interface. By avoiding reading a large amount of text, users may misunderstand the chatbot's message and finish with an unsuccessful result. The design of the chatbot dialogue should only have relevant information to help users with their tasks. Adding banal information could make the user feel dull to interact with the chatbot. 

\subsection*{Chatbot's Personality}

Chatbots' personality is what it makes a chatbot different. By adding extra information into the dialogue and having a matching language, the chatbot can have a distinctive personality. However, designers should be careful to not include too much additional information that makes the user feel bored or annoyed. The chatbot's personality should take into account the target audience for which it was designed. The personality should match with the users, therefore, how friendly, sarcastic or  joker should only depend on the people who will use the chatbot. Users like to interact with chatbots in a human way \cite{HeuristicsWebPage}. For this reason designers should keep into account how to engage users into the conversation. According to Scott \cite{HeuristicsWebPage}, there is a difference between the content (relevant information to help the user) and the medium (the chatbot's personality). Users need to be entertained and at the same time be helped.  The chatbot's success may be defined by how designers balance between entertainment and guidance. 

\subsection*{Chatbot's versatility}

Chatbots should be versatile enough to able to understand open questions and commands written by the user. If a user wants to use a command in the interaction with the chatbot, the dialogue should continue smoothly without breaking the flow. 

\subsection*{Chatbot's documentation}

Chatbots should be used without the need of extra information. How to use a chatbot should be a simple conversational flow. However, there are some users that like to deeply understand how the interface works and detect its limitations. A chatbot appropriate documentation should be precise and short. It should provide extra information about how the chatbot can help users to focus their tasks. Besides, it also "should include a concrete list of what steps can be carried out" and the critical points of the chatbot" \cite{HeuristicsWebPage}. Finally, the chatbot should provide an easy way to access the documentation.   


\medskip

\bibliographystyle{unsrt}%Used BibTeX style is unsrt
\bibliography{References.bib}

\end{document}