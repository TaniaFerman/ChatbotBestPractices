\documentclass[a4paper,10pt]{article}
\usepackage[utf8]{inputenc}

\title{Chatbot Design Principles}
\author{Maria Ferman}

\begin{document}

\maketitle

Nowadays, Software is getting more and more customized and at the same time, it is necessary to make software accessible for (to) users that do not have any technological background. Fortunately, chatbots can help with both issues. Chatbots allow users to interact and communicate with software by using their own words. Therefore, software can become easier to understand and use. Chatbots also allows to save the user preferences. By doing this, software can adapt to the interest of the users.   

\subsection*{Usability Heuristics}

\textbf{Match between the system and the real world}

Chatbots should use the same language and words as the user, and avoid to use technical and system-oriented terms. Chatbots can mimic how users normally speak, by designing a chatbot oriented to certain populations. Therefore, before the creation of the chatbot, it is necessary to have a solid understanding of the audience we seek to appeal to\cite{HeuristicsWebPage}.  

\subsection*{Why no to use chatbots}


%This document is an example of BibTeX using in bibliography management. Three items are cited: \textit{The \LaTeX\ Companion} book \cite{latexcompanion}, the Einstein journal paper \cite{einstein}, and the Donald Knuth's website \cite{knuthwebsite}. The \LaTeX\ related items are \cite{latexcompanion,knuthwebsite}. 

\medskip

\bibliographystyle{unsrt}%Used BibTeX style is unsrt
\bibliography{References.bib}

\end{document}