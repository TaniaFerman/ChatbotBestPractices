\documentclass[a4paper,10pt]{article}
\usepackage[utf8]{inputenc}
\usepackage{graphicx}
\title{Chatbot Design Principles}
\author{Maria Ferman}

\begin{document}

\maketitle

\section*{Project Description}

\textbf{Best Practices' Persona:}

\begin{center}
\includegraphics[scale=0.4]{/Users/MariaFerman/Desktop/BestPracticesPersona.png}
\end{center}


\textbf{Chatbot Best practices' Design:} The Best Practices are created to allow people like Chris to design an effective chatbot. These best practices give developers the most important aspects about the chatbot's design. Therefore, they will have an specific structure to follow of how to design  their chatbot.

\hfill \break

\textbf{Chatbot's Persona:}
\begin{center}
\includegraphics[scale=0.4]{/Users/MariaFerman/Desktop/PersonaProject.png}
\end{center}
\textbf{Chatbot:} The goal of the chatbot is to help people like Laura to create a meaningful visualization about her data. Laura can use natural language statements in order to communicate with the chatbot. In response, the chatbot will provide her with several visualization options.

%Alexey's feedback: (on Visualizations: to communicate or to explore the data) 
\textbf{Chatbot's tasks on Visualization:} The Visualization goal of the chatbot is to help users to analyse data by using visual context images (graphs). The chatbot might offer different options to visualize data. The options will be oriented to allow user to see patters, trends, and correlations that might be not detected by the use of data tables of excel sheets. Therefore, the chatbot will allow to easily see deeper insights on the user's data.  

\medskip

\bibliographystyle{unsrt}%Used BibTeX style is unsrt
\bibliography{References.bib}

\end{document}