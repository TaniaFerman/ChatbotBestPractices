\documentclass[a4paper,10pt]{article}
\usepackage[utf8]{inputenc}
\usepackage{graphicx}
\title{Towards forming conversational interface best practices for a visualization chatbot}
\author{Maria Ferman}

\begin{document}

\maketitle

\section*{Project Description}

The project consists of three main phases. The first is the design of a Chatbot's best practices. The second phase is the creation of a visualization chatbot in order to test the first phase. Finally the third section is the user case study. This study will be carried out to verify the chatbots' efficiency in interaction with the users in order to create effective visualizations.

\begin{figure}
\centering
\includegraphics[scale=0.4]{/Users/MariaFerman/Desktop/BestPracticesPersona.png}
\caption{Best Practices' Persona}
\label{FigureChris}
\end{figure}

\textbf{Chatbot Best Practices' Design:} The best practices are created to allow people like Chris (Figure ~\ref{FigureChris}) to design an effective chatbot. These best practices give developers the most important aspects about the chatbot's design. Therefore, from the beginning they will have a specific structure to follow of how to design  their chatbot.

\begin{figure}
\centering
\includegraphics[scale=0.4]{/Users/MariaFerman/Desktop/PersonaProject.png}
\caption{Chatbot's Persona}
\label{FigureLaura}
\end{figure}

\textbf{Chatbot:} The goal of the chatbot is to help people like Laura (Figure ~\ref{FigureLaura}) to create a meaningful visualization about their data. Laura can use natural language statements in order to communicate with the chatbot. In response, the chatbot will provide her with several visualization options.

%Alexey's feedback: (on Visualizations: to communicate or to explore the data) 
\textbf{Chatbot's Tasks on Visualization:} The Visualization goal of the chatbot is to help users to analyse data by using visual context images (graphs). The chatbot will offer different options to visualize data. The options will be oriented to allow users to see patterns, trends, and correlations that might be not detected by the use of data tables of excel sheets. Therefore, the chatbot will allow users to easily analyse their data in order to have deeper insights.  

\medskip

\bibliographystyle{unsrt}%Used BibTeX style is unsrt
\bibliography{References.bib}

\end{document}